\section{Introduction}

%This document provides instructions for submitting papers to the 50th
%International Symposium on microarchitecture (MICRO), 2017.  In an
%effort to respect the efforts of reviewers and in the interest of
%fairness to all prospective authors, we request that all submissions
%to MICRO 2017 follow the formatting and submission rules detailed
%below. Submissions that violate these instructions may not be reviewed,
%at the discretion of the program chairs, in order to maintain a review
%process that is fair to all potential authors.
%
%
%This document is itself formatted using the MICRO-50 submission format.
%The content of this document mirrors that of the submission
%instructions that appear on
%\href{http://www.microarch.org/micro50/Submission/}{this website}.
%
%All questions regarding paper formatting and submission should be directed
%to the program chairs.

\subsection{Problem}
\subsection{Key Features}
% Note that there are some changes from last year. 
%\begin{itemize} 
%\item Paper must be submitted in printable PDF format.
%\item Text must be in a minimum 10pt ({\bf not} 9pt) font.
%\item Papers must be at most 11 pages, not including references. 
%\item No page limit for references. 
%\item Each reference must specify {\em all} authors (no {\em et al.}). 
%\item Authors of {\em all} accepted papers will be required to give a
%lightning presentation (about 90s) and a poster in addition to the regular
%conference talk.
%\end{itemize} 
%
%\subsection{Paper Evaluation Objectives} 
%The committee will make every effort to judge each submitted paper on 
%its own merits. There will be no target acceptance rate. 
%We expect to accept a wide range of papers with appropriate expectations 
%for evaluation---while papers that build on significant past work 
%with strong evaluations are valuable, papers that open new areas with 
%less rigorous evaluation are equally welcome and especially encouraged. 
%Given the wide range of topics covered by MICRO, every effort will be 
%made to find expert reviewers, including providing the ability for authors' 
%to suggest additional reviewers. 

