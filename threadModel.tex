\section{Threat Model}
The adversary and the victim are part of different contexts (don't share the same address space). Adversary is trying to learn something about the victim, while not having any legitimate access to that piece of information and also has no special priveleges. We also assume that the adversary has no physical access to the machine, such as being able to probe memory or the bus, to monitor memory traffic. The only means for the adversary to learn any information about the victim's execution or data is by timing their own memory accesses to learn information about the victim's addresses. They can achieve the same by monitoring interference on different cache sets. The victim's address pattern that the adversary is able to learn from cache interference, has the potential to leak security critical information, such as bits of an encryption key. We focus on slowing down attacks that learn information via cache set contention \cite{p+p}. We do not protect against reuse based attacks, where the attacker simply relies on the fact that the previously accessed data will be cached and it's subsequent reference would be a cache hit, to learn information about the victim process. \cite{rfill,F+F}.  We do not consider other channels for information leakage such as EM \cite{} and power channels\cite{}.       
