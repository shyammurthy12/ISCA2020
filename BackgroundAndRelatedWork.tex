\section{Background and Related Work}
Cache based side channel attacks exploit the flow of information from keys into addresses. This is common in a lot of implementations of cryptographic algorithms, where we have table lookups, for which the index comes from certain bits of the key. Examples of the same are multiplier tables in the RSA algorithm, which is implemented as a lookup table and for which the index comes from bits of the key. \cite{rfill}
Cache based side channel attacks can be broadly classified into conflict based attacks and reuse based attacks. 
Conflict based attacks happen when the attacker and victim contend for the same cache set. The attacker exploits the deterministic nature of the mapping from addresses to sets, to learn information about the victim's address. Classic example of this kind of an attack is the Prime and Probe \cite{p+p}. Here, the attacker first primes or fills one or more sets of the cache with his data. Waits for the victim to run, and subsequently accesses the items to see if all the items are still cached. If any of the items see a longer access time, then the attacker knows which of his data items got evicted by the victim, and thereby learns information about the sets that saw contention from the victim. Our work seeks to mitigate these kinds of attacks, where the attack relies on learning the location of a memory line in the cache. 
Prior solutions to mitigate conflict based attacks in the L1 cache such as RPCache \cite{RPCache} and NewCache \cite{NewCache} require large table lookups, and additionally also require making changes to the address decoder logic within the cache, to tolerate the additional latency introduced, which our scheme avoids. CEASER \cite{ceaser} uses an encrypted line address to index the cache, whereas application of a similar solution in the L1 cache is expensive because the L1 cache access is on the critical path. Other approaches such as way partitioning that seek to partition the ways of a cache across different protection domains \cite{NoMo, DAWG} lead to under-utilization of cache space and become less feasible in the context of an L1 cache, where space becomes more of a premium. PLCache \cite{PLCache} requires locking of certain cache lines, which again leads to under-utilization of cache space.

Reuse Based attacks in contrast do not rely on the location of the memory line in the cache and instead only rely on the fact that a line that was accessed previoiusly will be cached and a subsequent reference would result in a cache hit. An example of such an attack is the Flush+Reload attack \cite{F+F}. Our work does not address these class of attacks. 
